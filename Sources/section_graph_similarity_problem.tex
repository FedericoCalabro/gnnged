\documentclass[../Thesis.tex]{subfiles}
\begin{document}
	
	\section{Graph Similarity Problem}
	\label{sec:graph_similarity_problem}
	
	Of particular interest is to understand whether two given graphs are similar or not, this question takes the name of \emph{graph similarity problem}. This problem could be of help in numerous domains and real world problems including  pattern recognition, computer vision, bioinformatics, social network analysis, and chemical informatics. In such fields, common problems can be modelled as graphs and comparing the structure properties of pair of those could be very beneficial. For instance, in bioinformatics, comparing protein interaction networks can reveal functional similarities between different proteins, while in social network analysis, it can help identify similar community structures within different social groups. Since graph similarity is very important, numerous metrics have been developed to measure graph similarity, each with its own strengths and limitations to take into account. In the following sections, we are going to explore several metrics commonly used to measure graph similarity: Graph Isomorphism, Graph Kernels, and Graph Edit Distance (GED). Each method will be discussed in terms of its fundamental concepts, applications, and limitations, with a focus the latter. Also, it is often desirable to retrieve the edit path from one graph to another in a straightforward manner to understand the specific transformations involved. However, we will focus solely on the similarity metrics and will not address the retrieval of edit paths.
	
	\subsection{Graph Isomorphism}
	
	In graph theory, graph isomorphism is one of the fundamental concepts used to determine if two graphs are structurally identical. Two graphs $G_1 = (V_1, E_1)$ and $G_2 = (V_2, E_2)$ are isomorphic if there is a bijection $f: V_1 \to V_2$ such that any two vertices $u$ and $v$ in $G_1$ are adjacent if and only if $f(u)$ and $f(v)$ are adjacent in $G_2$. Formally, $G_1$ and $G_2$ are isomorphic if:
	\[
	(u, v) \in E_1 \Leftrightarrow (f(u), f(v)) \in E_2
	\]
	Graph isomorphism metric provides in the a binary metric whether two graphs are identical in structure or not. Hence, it is limited because it does not quantify the degree of similarity at all. It is useful in scenarios where a binary outcome is desired. However, it is less useful in all the other cases where graphs are similar but not identical, as it cannot measure partial similarity or small structural differences.

	\subfile{../Tikz/tikz_graph_isomorphism}
	
	\subsection{Graph Kernels}
	
	A common solution in optimization theory when trying to separate two given dataset is to artificially increase their spatial dimension by using kernel tricks \cite{kerneltrick}. In the same way graph kernels transforms graphs into high-dimensional vectors where it is easier to compare them and exploit this mechanism to compute a similarity metric based on their structural attributes and properties. Common types of graph kernels include:
	
	\begin{itemize}
		\item \textbf{Random Walk Kernels}: Measure the similarity based on the number of matching random walks in both graphs.
		\item \textbf{Shortest Path Kernels}: Measure the similarity based on the distribution of shortest paths between pairs of nodes in each graph.
		\item \textbf{Weisfeiler-Lehman Kernels}: Measure the similarity utilizing an iterative node labeling algorithm to capture the neighborhood structure around each node.
	\end{itemize}
	
	Thus, structural information can be recovered in several different ways by utilizing graph kernels which can then be considered well-suited for use in machine learning algorithms where kernel tricks are commonly used to create algorithmic classificators. However, they can be computationally intensive if not carefully handled and also require careful tuning of parameters.
	
	\subfile{../Tikz/tikz_graph_kernels}
	
	\subsection{Graph Edit Distance (GED)}
	
	One of the most flexible and informative metric that measure the similarity between two graphs $G_1 = (V_1, E_1)$ and $G_2 = (V_2, E_2)$ is the \emph{graph edit distance} (GED). GED quantifies similarity by determining the minimum cost required to transform $G_1$ into $G_2$ by means of a series of atomic operations. These operations include but are not limited to vertex and edge insertions, deletions, and substitutions. The cost of each operation is determined by a predefined cost function which is usually $1$.
	
	Formally, let $\Sigma$ be the set of all possible edit operations, and let $c: \Sigma \to \mathbb{R}^+$ be a cost function that assigns a positive real number to each operation. The GED, which falls in the range [0, inf), is then given by:
	\[
	\text{GED}(G_1, G_2) = \min_{\sigma \in \Sigma^*} \sum_{o \in \sigma} c(o)
	\]
	where $\Sigma^*$ denotes the set of all finite sequences of operations from $\Sigma$, and $o$ represents an individual operation within a sequence $\sigma$.
	
	However, the computation of GED is known to be \emph{NP-HARD} \cite{complexityclasses}, which means that finding the exact minimum edit distance between two graphs is computationally intensive. Despite this, GED is preferred over other similarity metrics due to its flexibility and ability to provide a good measure of similarity even when the graphs are not identical.
	
	\subsubsection{Atomic Operations}
	
	The basic atomic operations in GED typically include:
	
	\begin{itemize}
		\item \textbf{Vertex Insertion}: Inserting a new vertex $v$ into the graph.
		\item \textbf{Vertex Deletion}: Deleting an existing vertex $v$ from the graph.
		\item \textbf{Vertex Substitution}: Replacing an existing vertex $v$ with another vertex $u$.
		\item \textbf{Edge Insertion}: Inserting a new edge $e = \{u, v\}$ into the graph.
		\item \textbf{Edge Deletion}: Deleting an existing edge $e = \{u, v\}$ from the graph.
		\item \textbf{Edge Substitution}: Replacing an existing edge $e = \{u, v\}$ with another edge $e' = \{u', v'\}$.
		\item \textbf{Node Relabelling}: Replacing the label $l$ of a vertex $v$ with another label $l'$.
	\end{itemize}
	
	To illustrate the concept of Graph Edit Distance (GED), consider the pair of graphs represented in [\autoref{fig:ged-graphs}]:
	\subfile{../Tikz/tikz_atomic_operations}
	In this example, graph $G_1$ has a vertex set $V_1 = \{A, B, C\}$ and an edge set $E_1 = \{\{A, B\}, \{A, C\}, \{B, C\}\}$, while graph $G_2$ has a vertex set $V_2 = \{A, B, C, D\}$ and an edge set $E_2 = \{\{A, B\}, \{A, C\}, \{B, C\}, \{B, D\}\}$. The transformation with the lowest cost from $G_1$ to $G_2$ involves inserting the vertex $D$ and inserting the edge $\{B, D\}$. If we assign a cost of 1 to each operation, the total cost (GED) is: $1+1=2$.
	
\end{document}
