\documentclass[../Thesis.tex]{subfiles}
\begin{document}
	\section{Discussion and Conclusions}
	\label{sec:discussion_and_conclusions}
	In this thesis, the Graph Edit Distance (GED) has been introduced and explained—a concept which is important in graph theory and which defines the cost of transformation of one graph into another. This is important considering that graphs are used in various fields including bioinformatics, social network analysis among others where proper and fast GED determination is crucial. Although new developments have been made in deep learning especially Graph Neural Networks (GNNs), most of the current methods do not have good generalization capabilities.
	
	The content of this paper shows that the current neural network-based solutions are quite innovative but they are associated with small and approximate datasets. It has been established that the usage of artificially generated datasets is promising; nevertheless, the current findings suggest that the size and density of such datasets remain limited, which hinders the development of models with the potential for generalization. More research should be directed towards the creation of bigger and more dense synthetic graphs similar to real graphs which would improve the performance of the models.
	
	In conclusion, it is possible to state that the current state of research on neural network-based GED computation has been significantly improved, yet, there is still a large discrepancy between the current state of research and the real-world needs for effective, efficient, and robust GED computation. In order to meet these challenges, it will be necessary to develop new and more complex models and, at the same time, to gather larger and more detailed datasets. This will be important for achieving the best results in the application of GED in numerous areas such as in drug development, analysis of social networks, and so on, and therefore enhance the effectiveness of solutions in these spheres.
\end{document}