\documentclass[../Thesis.tex]{subfiles}
\begin{document}
	\section{Discussion and Conclusions}
	\label{sec:discussion_and_conclusions}
	This thesis focuses on the Graph Edit Distance (GED), a concept in graph theory that quantifies the cost of transforming one graph into another. GED is especially useful in areas such as bioinformatics, social network analysis, and computer vision since graph similarity assessment is critical and time-consuming. Although the recent developments in deep learning, especially GNNs, the current approaches have limited generalization and fail in practical scenarios.
	
	In the course of this thesis, the following have been achieved. First, it provides a detailed survey of the current AI approaches for GED estimation and their advantages and disadvantages. This also involves a comprehensive analysis of models like GedGNN, SimGNN, TagSim, and GPN. Second, the thesis offers an enhanced repository with the latest code for testing these models. This repository has major changes such as changing the PyTorch version from 1. x to 2. x, which improves the speed and enables the user to test a wider range of scenarios. Third, a procedure for creating artificial datasets of graphs has been proposed, where the GED between any two graphs is known. This provides a useful source of information for future work and for the further refinement of the model.
	
	But the main finding of this work is that all the models, even with their advancements, fail to perform well on the out-of-distribution data. This is the most important observation since generalization is crucial for practical use. The synthetic datasets employed in the study are promising, but at the moment they are still relatively small and sparse to enhance the model’s accuracy. In the future, it will be crucial to use larger and more detailed synthetic datasets that will resemble real-world graphs to some extent to get better results.
	
	In conclusion, it can be stated that despite the achievements made in the development of neural network-based GED computation, the current methods are still far from meeting the requirements of practical applications. Further studies should aim at enhancing the complexity of the models as well as the quality of the data to enhance the model’s ability to generalize. These advancements are important for the improvement of GED in various areas such as drug discovery, social network analysis and many more.
\end{document}