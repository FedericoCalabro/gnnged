\documentclass[../Thesis.tex]{subfiles}
\begin{document}
	
	\section{Graph Data Structure}
	\label{sec:graph_data_structure}
	
	A \emph{graph} $G$ [\autoref{fig:example-graph}] is a nonlinear data structure consisting of a set of vertices and arcs, where arcs connect pairs of vertices in the set. Graphs are widely used to represent relationships between entities and play a significant role in the development of fields like Computer Science, Optimization, Chemistry and others. They are a pillar in network-based systems modeling such as social media, biological networks, and transportation systems, being a crucial tool for analyzing and solving complex problems. Use cases of graphs can be found in the actual world, for example in recommendation systems, routing and navigation algorithms like GPS, optimization problems and resource allocation (also known as transportation problems).
	
	\subfile{../Tikz/tikz_graph_example}
	
	A graph can be formally defined as a tuple $G = (V, E)$, where:
	\begin{itemize}
		\item $V$ is a finite set of vertices where each represents an entity or a data point. The set $V$ is often denoted as $V = \{v_1, v_2, \ldots, v_n\}$ where $n$ is the number of vertices.
		\item $E$ is a set of edges, where each edge is an unordered pair of distinct vertices from $V$. Thus, $E \subseteq \{\{u, v\} \mid u, v \in V \text{ and } u \neq v\}$. Edges represents the existing relationship between two vertices in the set.
	\end{itemize}
	
	For instance, graph depicted in \autoref{fig:example-graph} can be formally defined as a tuple \( G = (V, E) \), where:
	\begin{itemize}
		\item \( V \) is the set of vertices, \( V = \{A, B, C, D\} \)
		\item \( E \) is the set of edges, \( E = \{(A, B), (A, C), (B, C), (B, D)\} \)
	\end{itemize}
	
	\subsection{Types of Graphs}
	There exist different categories of graphs depending on their properties, including:
	
	\begin{itemize}
		\item \textbf{Directed Graph} [\autoref{fig:directed-graph}]: also known as digraph, is the case where the direction is indicated on the edges, representing \( G \) as an ordered pair \( (u, v) \) where \( u, v \in V \) and \( u \neq v \). It's applied in a range of areas, including web page ranking, where links between pages have a set direction, and citation networks, where one paper references another.
		\subfile{../Tikz/tikz_directed_graph}
		
		\item \textbf{Undirected Graph} [\autoref{fig:undirected-graph}]: the edges do not have a direction, represented as an unordered pair \( \{u, v\} \) where \( u, v \in V \) and \( u \neq v \). This kind of graph is commonly used to model networks where the connections of two nodes are mutual, indicating that relationship is valid in both senses.
		\subfile{../Tikz/tikz_undirected_graph}
		
		\item \textbf{Weighted Graph} [\autoref{fig:weighted-graph}]: in this graph edges have a weight (or cost) related, represented as a function \( w: E \to \mathbb{R} \) where \( w(e) \) is the weight of edge \( e \in E \). This is particularly useful in transportation networks where the weights can express distances, the time spent traveling from one point to another, or costs associated with the displacement.
		\subfile{../Tikz/tikz_weighted_graph}
		
		\item \textbf{Simple Graph} [\autoref{fig:simple-graph}]: is a graph without loops (there doesn't exist a path from a vertex to itself) and has no multiple edges (the same pair of vertices is not connected more than once). Simple graphs are the most basic type of graph existing, with straightforward structures that make them easy to handle. They are often used for modeling basic networks to maintain a clear design and facilitate the analysis of the structure and the understanding of network properties. 
		\subfile{../Tikz/tikz_simple_graph}
		
		\item \textbf{Complete Graph} [\autoref{fig:complete-graph}]: is a graph in which every vertex is connected with all the other vertices in the set. Formally, a complete graph on \( n \) vertices, denoted as \( K_n \), has \( E = \{ \{u, v\} \mid u, v \in V, u \neq v \} \). They are widely used in cases where is necessary maximum connectivity, such as some network topologies and combinatorial optimization problems.
		\subfile{../Tikz/tikz_complete_graph}
		
		\item \textbf{Bipartite Graph} [\autoref{fig:bipartite-graph}]: a graph whose vertices are separable into two disjoint sets \( U \) and \( W \) in a way that an edge only connects a vertex from \( U \) with a vertex from \( W \). Bipartite graphs are useful for modeling relationships between objects from two different classes. For example, in the context of job assignment, vertices in \( U \) can symbolize jobs and vertices in  \( W \) workers.  Edges will indicate which job is assigned to each worker.
		\subfile{../Tikz/tikz_bipartite_graph}
		
		\item \textbf{Multigraph} [\autoref{fig:multigraph}]: it is a graph where multiple edges occur between the same pair of vertices. Formally, \( G = (V, E) \) where \( E \)  is a multiset of unordered pairs of vertices. Multigraphs are often used for modeling networks where multiple relationships or interactions exist for the same pair of vertices. For example, it is known that in transportation networks exists multiple routes or connections between two locations.
		\subfile{../Tikz/tikz_multigraph}
		
		\item \textbf{Cyclic Graph} [\autoref{fig:cyclic-graph}]: is the one that contains at least a cycle, being a cycle a path where every vertex on it is reachable from itself. Cyclic graphs are employed to model processes or systems where feedback loops are present. For example, it could be said that in certain biological systems or recurrent neural networks, loops represent the actions or the recurrent connections respectively.
		\subfile{../Tikz/tikz_cyclic_graph}
		
		\item \textbf{Acyclic Graph} [\autoref{fig:acyclic-graph}]: An acyclic graph is a graph without loops. A direct acyclic graph (DAG) is a directed graph without loops. Acyclic graphs, specially DAGs, are mainly used for modeling cases as task scheduling, where dependencies should not form cycles. In this kind of situation, a task will start only if all its prerequisites are completed. The absence of loops ensures that there aren't circular dependencies.
		\subfile{../Tikz/tikz_acyclic_graph}
		
	\end{itemize}
	
	\subsection{Graph Representation}
	There are several manners to represent a graph, including:
	
	\begin{itemize}
		\item \textbf{Adjacency Matrix} [\autoref{fig:adjacency-matrix}]: An adjacency matrix \( A \) corresponding to a graph \( G = (V, E) \) is a binary square matrix of size \( |V| \times |V| \)  that expresses the existence of a relationship between a pair of vertices. The value of \( A_{ij} \) is 1 if there is an edge connecting vertices \( v_i \) and \( v_j \), and 0 otherwise. This structure is particularly convenient for dense graphs where the number of edges is nearest to the limit of possible edges. It facilitates efficiency in the querying process to know if an edge exists and is easy to implement for algorithms that require constant monitoring of the edge's presence. Even so, the space complexity is \( O(|V|^2) \), which is a problem for large graphs with many vertices.
		\subfile{../Tikz/tikz_adjacency_matrix}
		
		\item \textbf{Adjacency List} [\autoref{fig:adjacency-list-undirected}]:  An adjacency list is a collection of lists where each one corresponds to a vertex and contains all the adjacent vertices. Having a graph  \( G = (V, E) \), the adjacency list could be implemented as a list array \( \{L_1, L_2, \ldots, L_n\} \) and every \( L_i \) will include all \( v_i \)'s neighbors. These structures are more efficient for sparse graphs due to the number of edges is much smaller than the number of possible edges. The use of an adjacency list facilitates the work of graph algorithms such as breadth-first search (BFS) and depth-first search (DFS), where only the more important neighbors need to be visited.
		\subfile{../Tikz/tikz_adjacency_list}
		
	\end{itemize}
	
	\subsection{Properties of Graphs}
	Graphs have some outstanding properties that help to analyze them and determine the best application for each type, including:
	
	\begin{itemize}
		\item \textbf{Degree} [\autoref{fig:degree-example}]: The degree of a vertex is the number of vertices in the graph that incident on it. Formally, for a vertex \( v \) in a graph \( G = (V, E) \), the degree \( \deg(v) \) is the number of vertices connected to it. In the case of direct graphs, the in-degree stands for the number of incoming edges, and the out-degree stands for the number of outgoing edges. Vertices with high degrees generally play a crucial role in a graph, indicating significant or highly connected nodes.
		
		\subfile{../Tikz/tikz_degree_example}
		
		\item \textbf{Connectivity} [\autoref{fig:connectivity-example}]: Connectivity refers to the way nodes are connected within the graph. A graph is called connected if there exists a path between every pair of vertices, in other words, each vertex is reachable from any other vertex in the graph. This is a key property for understanding reliability and network robustness and ensures that all nodes can communicate directly or indirectly between them.
		\subfile{../Tikz/tikz_connectivity_example}
		
		\item \textbf{Centrality} [\autoref{fig:centrality-example}]: The centrality measures are employed to identify the most significant vertices in a graph. Some of the most important metrics are degree centrality, which quantifies the direct connections to a vertex; closeness centrality, which evaluates the velocity of a node to reach another; and betweenness centrality, which assesses how many times a vertex acts as a bridge in the closeth path between a pair of vertices. These metrics offer distinct points of view about the importance and influence of a node in the net.
		\subfile{../Tikz/tikz_centrality_example}
		
		\item \textbf{Clustering Coefficient} [\autoref{fig:clustering-coefficient-example}]: The clustering coefficient of a vertex measures how connected the neighbors of that vertex are to each other. A high clustering coefficient suggests a community closely linked in the graph. In mathematics terms, it is defined as the proportion between the real edges and the possible edges among the vertex neighbors.
		\subfile{../Tikz/tikz_clustering_coefficient_example}
		
		\item \textbf{Graph Diameter} [\autoref{fig:graph-diameter-example}]: The diameter of a graph is the length of the longest shortest paths between any pair of vertices. This metric indicates the "spread" of the graph and helps to understand how distant the vertices are, considering the minimum distance that connects them.
		\subfile{../Tikz/tikz_graph_diameter_example}
		
		\item \textbf{Graph Density} [\autoref{fig:graph-density-example}]: The density of a graph is defined as the proportion between the number of existing edges and the maximum possible edges among the vertices. In an undirected graph with \( n \) vertices, the total of possible edges is \( \frac{n(n-1)}{2} \). The density indicates how close the graph is to completeness, it means,  how close it is to having all possible edges.
		\subfile{../Tikz/tikz_graph_density_example}
		
	\end{itemize}
	
\end{document}