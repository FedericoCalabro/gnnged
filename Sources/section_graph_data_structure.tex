\documentclass[../Thesis.tex]{subfiles}
\begin{document}
	
	\section{Graph Data Structure}
	\label{sec:graph_data_structure}
	
	A \emph{graph} $G$ [\autoref{fig:example-graph}] is a non-linear data structure consisting of vertices and edges. The vertices are sometimes also referred to as nodes and the edges are arcs that connect any two nodes in the graph. Graphs are used to represent relationships between different entities and have applications in many fields including Computer Science, Physics, Biology, Chemistry, Optimization Theory, Social Sciences, and many more. They are fundamental in modeling networks such as social networks, communication networks, biological networks, and transportation systems, making them indispensable tools for analyzing and solving complex problems. In everyday tasks, graphs can represent relationships in recommendation systems, routing algorithms in GPS navigation, workflow optimization, and resource allocation in various industries.
	
	\subfile{../Tikz/tikz_graph_example}
	
	A graph is formally defined as a tuple $G = (V, E)$, where:
	\begin{itemize}
		\item $V$ is a finite set of vertices (or nodes). Each vertex represents an entity or a data point, and the set of vertices $V$ is often denoted as $V = \{v_1, v_2, \ldots, v_n\}$ where $n$ is the number of vertices.
		\item $E$ is a set of edges, where each edge is an unordered pair of distinct vertices from $V$. Thus, $E \subseteq \{\{u, v\} \mid u, v \in V \text{ and } u \neq v\}$. Each edge signifies a relationship or connection between the pair of vertices it links.
	\end{itemize}
	
	For instance The graph depicted in \autoref{fig:example-graph} can be formally defined as a tuple \( G = (V, E) \), where:
	\begin{itemize}
		\item \( V \) is the set of vertices, \( V = \{A, B, C, D\} \)
		\item \( E \) is the set of edges, \( E = \{(A, B), (A, C), (B, C), (B, D)\} \)
	\end{itemize}
	
	\subsection{Types of Graphs}
	Graphs can be classified into various types based on their properties, including:
	
	\begin{itemize}
		\item \textbf{Directed Graph} [\autoref{fig:directed-graph}]: A directed graph (or digraph) is a graph in which every edge has a direction, represented as an ordered pair \( (u, v) \) where \( u, v \in V \) and \( u \neq v \). It is used in various applications such as web page ranking, where links from one page to another have a specific direction, and citation networks, where one paper cites another.
		\subfile{../Tikz/tikz_directed_graph}
		
		\item \textbf{Undirected Graph} [\autoref{fig:undirected-graph}]: An undirected graph is a graph in which the edges do not have a direction, represented as an unordered pair \( \{u, v\} \) where \( u, v \in V \) and \( u \neq v \). This type of graph is commonly used to model social networks where the connections (friendships) are mutual, indicating that if one person is friends with another, the reverse is also true.
		\subfile{../Tikz/tikz_undirected_graph}
		
		\item \textbf{Weighted Graph} [\autoref{fig:weighted-graph}]: A weighted graph is a graph in which each edge has an associated weight or cost, represented as a function \( w: E \to \mathbb{R} \) where \( w(e) \) is the weight of edge \( e \in E \). This is particularly useful in transportation networks where the weights can represent distances, travel times, or costs associated with traveling between locations.
		\subfile{../Tikz/tikz_weighted_graph}
		
		\item \textbf{Simple Graph} [\autoref{fig:simple-graph}]: A simple graph is a graph that has no loops (edges connecting a vertex to itself) and no multiple edges (more than one edge connecting the same pair of vertices). Simple graphs are the most basic type of graph, with straightforward structures that make them easy to analyze. They are used in many basic network models to simplify the analysis and understand the fundamental properties of the network.
		\subfile{../Tikz/tikz_simple_graph}
		
		\item \textbf{Complete Graph} [\autoref{fig:complete-graph}]: A complete graph is a graph in which there is exactly one edge between each pair of distinct vertices. Formally, a complete graph on \( n \) vertices, denoted as \( K_n \), has \( E = \{ \{u, v\} \mid u, v \in V, u \neq v \} \). Complete graphs are used in scenarios where maximum connectivity is required, such as in certain network topologies and in combinatorial optimization problems.
		\subfile{../Tikz/tikz_complete_graph}
		
		\item \textbf{Bipartite Graph} [\autoref{fig:bipartite-graph}]: A bipartite graph is a graph whose vertices can be divided into two disjoint sets \( U \) and \( W \) such that every edge connects a vertex in \( U \) to a vertex in \( W \). Bipartite graphs are used to model relationships between two different classes of objects. For example, in job assignments, vertices in \( U \) can represent jobs, and vertices in \( W \) can represent workers, with edges indicating which workers are assigned to which jobs.
		\subfile{../Tikz/tikz_bipartite_graph}
		
		\item \textbf{Multigraph} [\autoref{fig:multigraph}]: A multigraph is a graph which allows multiple edges between the same pair of vertices. Formally, \( G = (V, E) \) where \( E \) is a multiset of unordered pairs of vertices. Multigraphs are used to model scenarios where multiple relationships or interactions can exist between entities. For example, in transportation networks, multiple routes or connections can exist between the same pair of locations, and these multiple edges can represent different routes or modes of transport.
		\subfile{../Tikz/tikz_multigraph}
		
		\item \textbf{Cyclic Graph} [\autoref{fig:cyclic-graph}]: A cyclic graph is a graph that contains at least one cycle, where a cycle is a path of edges and vertices wherein a vertex is reachable from itself. Cyclic graphs are used to model processes or systems where feedback loops are present. For example, in certain biological systems or in recurrent neural networks, cycles can represent the feedback mechanisms or recurrent connections.
		\subfile{../Tikz/tikz_cyclic_graph}
		
		\item \textbf{Acyclic Graph} [\autoref{fig:acyclic-graph}]: An acyclic graph is a graph with no cycles. A directed acyclic graph (DAG) is a directed graph with no directed cycles. Acyclic graphs, especially directed acyclic graphs (DAGs), are used in scenarios such as scheduling tasks, where dependencies must not form cycles. In such cases, a task can only start once all its prerequisite tasks are completed, and the absence of cycles ensures that there are no circular dependencies.
		\subfile{../Tikz/tikz_acyclic_graph}
		
	\end{itemize}
	
	\subsection{Graph Representation}
	Graphs can be represented in various ways, including:
	
	\begin{itemize}
		\item \textbf{Adjacency Matrix} [\autoref{fig:adjacency-matrix}]: An adjacency matrix \( A \) for a graph \( G = (V, E) \) is a square matrix of size \( |V| \times |V| \). The entry \( A_{ij} \) is 1 if there is an edge between vertices \( v_i \) and \( v_j \), and 0 otherwise. This representation is particularly useful for dense graphs, where the number of edges is close to the maximum possible number of edges. It allows for efficient querying of edge existence and is easy to implement for algorithms that require frequent checks of edge presence. However, the space complexity is \( O(|V|^2) \), which can be prohibitive for large graphs with many vertices.
		\subfile{../Tikz/tikz_adjacency_matrix}
		
		\item \textbf{Adjacency List} [\autoref{fig:adjacency-list-undirected}]: An adjacency list is a collection of lists or arrays where each list corresponds to a vertex and contains all the vertices adjacent to that vertex. For a graph \( G = (V, E) \), the adjacency list can be represented as an array of lists \( \{L_1, L_2, \ldots, L_n\} \), where each \( L_i \) contains the neighbors of vertex \( v_i \). This representation is more space-efficient for sparse graphs, where the number of edges is much smaller than the number of possible edges. It facilitates efficient traversal operations such as breadth-first search (BFS) and depth-first search (DFS), where only the relevant neighbors of each vertex need to be examined.
		\subfile{../Tikz/tikz_adjacency_list}
		
	\end{itemize}
	
	\subsection{Properties of Graphs}
	Graphs possess various properties that help in their analysis and application. Some of these properties include:
	
	\begin{itemize}
		\item \textbf{Degree} [\autoref{fig:degree-example}]: The degree of a vertex is the number of edges incident to it. For a vertex \( v \) in a graph \( G = (V, E) \), the degree \( \deg(v) \) is the count of edges connected to \( v \). In directed graphs, the in-degree represents the number of incoming edges and the out-degree represents the number of outgoing edges. High-degree vertices often play a crucial role in the graph, indicating significant or highly connected nodes.
		\subfile{../Tikz/tikz_degree_example}
		
		\item \textbf{Connectivity} [\autoref{fig:connectivity-example}]: Connectivity refers to how well nodes are interconnected within a graph. A graph is said to be connected if there is a path between every pair of vertices. This property is vital for understanding network reliability and robustness, as it ensures that all nodes can communicate or reach each other directly or indirectly.
		\subfile{../Tikz/tikz_connectivity_example}
		
		\item \textbf{Centrality} [\autoref{fig:centrality-example}]: Centrality measures are used to identify the most important vertices within a graph. Different types of centrality include degree centrality, which counts the number of direct connections a vertex has; closeness centrality, which measures how quickly a vertex can access other vertices; and betweenness centrality, which quantifies how often a vertex acts as a bridge along the shortest path between other vertices. These measures provide various insights into the roles and influence of vertices in a network.
		\subfile{../Tikz/tikz_centrality_example}
		
		\item \textbf{Clustering Coefficient} [\autoref{fig:clustering-coefficient-example}]: The clustering coefficient of a vertex measures the extent to which neighbors of the vertex are also connected to each other. A high clustering coefficient indicates a tightly-knit community within the graph. In mathematical terms, it is calculated as the ratio of the number of actual edges to the number of possible edges among the neighbors of a vertex.
		\subfile{../Tikz/tikz_clustering_coefficient_example}
		
		\item \textbf{Graph Diameter} [\autoref{fig:graph-diameter-example}]: The diameter of a graph is the length of the longest shortest path between any pair of vertices. This metric provides an indication of the "spread" of the graph and helps in understanding how far apart vertices can be in terms of the shortest path distance.
		\subfile{../Tikz/tikz_graph_diameter_example}
		
		\item \textbf{Graph Density} [\autoref{fig:graph-density-example}]: Graph density is defined as the ratio of the number of edges in the graph to the number of possible edges between vertices. For a graph with \( n \) vertices, the maximum number of edges is \( \frac{n(n-1)}{2} \) in an undirected graph. Density provides a measure of how close the graph is to being a complete graph, where all possible edges are present.
		\subfile{../Tikz/tikz_graph_density_example}
		
	\end{itemize}
	
\end{document}