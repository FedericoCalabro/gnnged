\documentclass[../Thesis.tex]{subfiles}
\begin{document}
	\section{Introduction}
	\label{sec:introduction}
	
	Graphs are fundamental structures in computer science and mathematics that model relationships between entities. They consist of vertices (or nodes) and edges (or links) that connect pairs of vertices. This simple yet powerful representation can capture a wide variety of real-world scenarios, providing a versatile tool for analyzing complex systems. For instance, social networks can be represented as graphs where nodes denote individuals, and edges represent their connections or interactions, enabling the study of social dynamics, influence, and community formation. In biology, protein-protein interaction networks, neural networks of the brain, and ecological networks can all be modeled as graphs, facilitating the understanding of biological processes, brain functionality, and ecosystem interdependencies. Similarly, in transportation, cities can be nodes and roads or flights can be edges, creating a network that facilitates route optimization, urban planning, and logistical efficiency. Graphs can also model communication networks, where devices are nodes and connections are edges, allowing for analysis of data flow, network robustness, and optimization of resource allocation.
	
	Graph theory provides a rich framework for analyzing these structures, with properties such as connectivity, centrality, and clustering coefficient helping to understand the underlying patterns and behaviors within the network. Connectivity measures how well the nodes are connected, centrality identifies the most important nodes within the graph, and clustering coefficient gives insight into the degree to which nodes tend to cluster together. Additionally, other important graph properties include graph diameter, which measures the longest shortest path between any two nodes, and graph density, which indicates the level of interconnectedness in the network. These properties help in uncovering critical information about the structure and function of the graph, enabling more effective analysis and decision-making.
	
	The Graph Edit Distance (GED) problem is a critical measure in graph theory, providing a similarity metric between two graphs. GED quantifies how many operations (such as insertions, deletions, and substitutions of nodes and edges) are required to transform one graph into another. This measure is invaluable for various applications, including bioinformatics, where it can compare molecular structures to identify potential drug candidates or understand evolutionary relationships. In computer vision, GED is crucial for object recognition, where the structural similarity between graphical representations of different objects must be assessed to identify and classify them accurately. Other graph similarity measures include graph isomorphism, which checks for exact structural similarity, and subgraph isomorphism, which identifies if one graph is a subgraph of another, useful for pattern matching and searching within larger networks.
	
	Knowing the exact GED between two graphs can provide profound insights. For example, in bioinformatics, understanding the similarity between different molecular structures can lead to the discovery of new drugs and therapeutic targets by revealing structural patterns that correlate with biological activity. In social network analysis, GED can help detect communities or clusters of users with similar interaction patterns, aiding in the identification of influential individuals, the spread of information, or the formation of social groups. Moreover, in pattern recognition and image analysis, GED can assist in identifying objects and understanding their structural relationships, enhancing the accuracy and reliability of automated systems. The ability to quantify the similarity between graphs allows for more precise and meaningful comparisons, driving innovations and improvements across these fields.
	
	However, computing the exact GED is notoriously difficult due to its high computational complexity. The problem is NP-hard, meaning that the time required to solve it grows exponentially with the size of the graphs, making it computationally prohibitive for large graphs. This involves an exhaustive search over all possible edit paths, which is impractical for real-world applications. Various heuristics and approximation algorithms have been proposed, but they often struggle to balance accuracy and computational efficiency, leading to trade-offs that can impact the reliability of the results. The NP-hard class encompasses problems that are at least as hard as the hardest problems in NP, and no known polynomial-time algorithm can solve them. This inherent difficulty underscores the challenge of computing GED and the necessity for developing efficient approximation methods that can provide accurate results within reasonable timeframes.
	
	Neural networks, a cornerstone of modern machine learning, have revolutionized numerous fields by providing robust methods for handling complex, high-dimensional data. A neural network is a series of algorithms that attempt to recognize underlying relationships in a set of data through a process that mimics the way the human brain operates. They consist of interconnected layers of nodes, or neurons, each capable of processing inputs and producing outputs. Neural networks are trained using large datasets where they adjust their internal parameters based on the error between the predicted outputs and the actual outputs. This training process involves forward propagation, where inputs are passed through the network to generate outputs, and backpropagation, where the error is propagated back through the network to update the weights, thereby improving the model's accuracy. Neural networks have been successfully applied to a wide range of tasks, including image and speech recognition, natural language processing, and more recently, graph-structured data analysis, demonstrating their versatility and effectiveness.
	
	Graph Neural Networks (GNNs) are a specialized type of neural network designed to work directly with graph-structured data. GNNs aim to leverage the graph's inherent structure by performing convolution operations over the nodes and edges, capturing both local and global graph properties. This makes them well-suited for various tasks, including node classification, link prediction, and graph classification. Given their ability to learn complex patterns and representations, GNNs hold promise for approximating the GED. GNNs operate by iteratively updating the representation of each node based on its neighbors, effectively capturing the dependencies and relationships within the graph. This iterative process enables GNNs to learn hierarchical representations that are crucial for understanding and analyzing graph-structured data, allowing for more accurate and meaningful predictions in various applications.
	
	This thesis reviews a range of key articles to explore the current state-of-the-art methods in GED computation, encompassing both neural network-based approaches and traditional methods. The reviewed works span various innovative strategies, each attempting to tackle the challenges of GED computation from different angles. By critically analyzing these methods, this review aims to identify their strengths and limitations, offering insights into potential improvements. The seminal paper on SimGNN \cite{simgnn__a_neural_network_approach_to_fast_graph_similarity_computation} serves as a foundation for many subsequent works, introducing a neural network-based approach to GED computation that has inspired numerous advancements. More recent works, such as GedGNN \cite{computing_graph_edit_distance_via_neural_graph_matching}, continue to push the boundaries of what is possible, integrating novel techniques and improving upon previous methods.
	
	Improving GED computation methods is crucial for enhancing the performance of numerous applications that rely on graph similarity measures. For instance, more efficient and accurate GED computation can lead to breakthroughs in drug discovery by enabling faster and more precise comparison of molecular structures, facilitating the identification of new compounds with therapeutic potential. In social network analysis, it can facilitate the detection of more accurate community structures, leading to better understanding and management of social dynamics, improving the effectiveness of interventions and policy decisions. In computer vision, improved methods can enhance object recognition systems, making them more reliable and efficient, which is vital for applications ranging from autonomous vehicles to security systems. The implications of better GED computation extend to numerous domains, highlighting the importance of continued research and development in this area to unlock new possibilities and advancements.
	
	As we delve into the review of these articles, the goal is to provide a comprehensive overview of the advancements in GED computation. By highlighting innovative strategies and pinpointing areas for further research, this thesis aims to contribute to the ongoing efforts to refine and enhance GED computation methods. We will reproduce the results of key recent papers, such as the one proposing GedGNN, to validate their findings. Additionally, this thesis will offer critical and constructive advice on aspects such as code quality, the fairness of presented results, and the limitations of the datasets used. We will discuss issues like poor dataset quality and propose solutions, including artificial dataset generation and the development of neural networks that can be tested on any dataset, ensuring a fairer evaluation. This comprehensive review aims to bridge the gap between existing methods and the potential for new, more effective techniques, ultimately contributing to the broader field of graph theory and its myriad applications. By providing a thorough analysis and constructive feedback, this thesis seeks to guide future research and development efforts, paving the way for advancements that will enhance the accuracy, efficiency, and applicability of GED computation methods in various fields.
	
	
\end{document}
